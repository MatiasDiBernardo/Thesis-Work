\subsection{Modelos de TTS}
Describir la parte de TTS basados en modelos de difusión hasta llegar a F5 TTS.

\subsection{Cadenas de pre procesamiento}
En los últimos años se han desarrollado numerosas cadena de procesamiento, todas con diferentes configuraciones y particularidades. Para ilustrar las diferencias en las diferentes etapas se conforma la Tabla \ref{tab:preproc-comparison}, donde se comparan diferencias de modelos, configuraciones y criterios en el desarrollo de cadena de pre procesamiento automático para la creación de datasets.

\begin{table*}[ht]
\centering
\caption{Comparación entre diferentes etapas en una cadena de pre procesamiento para TTS.}
\label{tab:preproc-comparison}
\small
\begin{tabularx}{\textwidth}{@{} >{\raggedright\arraybackslash}X 
                                >{\raggedright\arraybackslash}X 
                                >{\raggedright\arraybackslash}X 
                                >{\raggedright\arraybackslash}X 
                                >{\raggedright\arraybackslash}X @{}}
\toprule
\makecell[b]{\textbf{Nombre}\\\textbf{del estudio}} & 
\makecell[b]{\textbf{Algoritmo de}\\\textbf{Denoising}} & 
\makecell[b]{\textbf{Voice Activity}\\\textbf{Detection}} & 
\makecell[b]{\textbf{Estimador MOS}\\\textbf{y umbral}} & 
\makecell[b]{\textbf{Sistema TTS}\\\textbf{evaluados}} \\
\midrule
AutoPrep - (\cite{autoprep}) & BSRRN & TDNN & DNS MOS: 2.4 & DurIAN TTS \\

Text-to-Speech in the wild - (\cite{tts_wild}) & Demucs & Whisper X Pipeline & Nisqa: 3 & GradTTS y VITS \\

WeNeetSpeech - (\cite{pipeline_tts2}) & MBTFNet & Rezamblyzer & DNS MOS: 3.6, 3.8, 4 & VALL-E y NS2 \\

SCEP - (\cite{pipeline_data1}) & U-Net & Casual DNN & Usan SNR y PESQ & No evalúa \\

Muyan TTS - (\cite{pipeline_tts1}) & FRCRN y VoiceFixer & No usa & Nisqa: 3.8 & FireRedTTS y CozyVoice2 \\

Emilia - (\cite{emilia}) & UVR-MDX-N et Inst & Silero VAD & DNS MOS: 3 & VoiceBox \\

\bottomrule
\end{tabularx}
\end{table*}
