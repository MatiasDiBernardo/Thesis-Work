\begin{centering}
\Large{\textbf{AGRADECIMIENTOS}}

\end{centering}

Se propone incluir este apartado, donde se debe agradecer primeramente a las autoridades de la Universidad, al coordinador de la carrera, al tutor y a los docentes implicados en el desarrollo de la investigación. Seguidamente agradecer a familiares o a aquellas personas que se quiera. También puede incluirse en la siguiente hoja una dedicatoria personal. A modo de ejemplo el contenido podría ser:

“En primer lugar dar gracias a la Universidad Nacional de Tres de Febrero (UNTREF), a su Rector Lic. Anibal Jozami, a todo su personal docente y no docente. Por promover un espacio ideal para el desarrollo de ideas y nuevos pensamientos y brindar a todos y cada uno de los alumnos, de esta casa de altos estudios, todos los recursos que esta institución dispone.   
Esta investigación no hubiera sido posible sin una formación académica acorde, por este motivo debo extender mi agradecimiento a los docentes de la carrera de Ingeniería de Sonido de la UNTREF, a su coordinador Ing. Alejandro Bibondo, que siendo la primera carrera de estas características del país, es muy importante contar con un cuerpo docente afín a las exigencias que este desafío propone, prestando su dedicación y vocación de enseñar. 
Un especial agradecimiento por la participación de esta tesis a la tutora Ing. Nombre Apellido, que supo transmitirme sus conocimientos y ayudarme a organizarme y fijarme un rumbo concreto y delineado, disponiendo desmedidamente de su tiempo. 
Por otra parte, quisiera hacer una mención especial al Ing. Hernan San Martin, que permitió el uso de las instalaciones de su laboratorio para poder trabajar y la disposición de todos sus recursos para que dicha investigación se realizara en tiempo y forma. 
Por último y no menos importante, quiero dar un afectuoso y cálido agradecimiento a mi familia…”

\newpage
