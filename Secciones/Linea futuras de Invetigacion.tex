Este trabajo pretende contribuir a la unificación de criterios en el diseño y evaluación de cadenas de preprocesado, lo que facilitará la identificación de las configuraciones más adecuadas para distintos casos de uso. Entre las líneas futuras de investigación se destacan, en particular, el desarrollo de modelos de denoising o speech enhancement personalizados: mediante técnicas de adaptación, dichos modelos buscarían aproximar conjuntos de grabaciones diversas hacia el dominio acústico del corpus con el que fue entrenado el modelo TTS, aumentando así la robustez y la fidelidad de la síntesis en condiciones reales.